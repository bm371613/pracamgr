\documentclass{pracamgr}

\usepackage{polski}

\usepackage[latin2]{inputenc}
\usepackage[T1]{fontenc}

\author{Bartosz Marcinkowski}

\nralbumu{319476}

\title{Przetwarzanie strumieniowe i analiza w czasie rzeczywistym}
\tytulang{Stream processing and realtime analytics}

\kierunek{Informatyka}

\opiekun{prof. Jana Madeya}

\date{Maj 2016}

\dziedzina{11.3 Informatyka}

\klasyfikacja{
D. todo \\
  D.127. todo \\
  D.127.6. todo}

\keywords{
	strumienie danych,
	systemy czasu rzeczywistegom,
	narz�dzia do analizy danych
}

% Tu jest dobre miejsce na Twoje w�asne makra i~�rodowiska:
\newtheorem{defi}{Definicja}[section]

% koniec definicji

\begin{document}
\maketitle

\begin{abstract}
Konieczno�� przetwarzania szybko nap�ywaj�cych danych wymusi�a powstanie system�w przetwarzania strumieniowego, kt�re s� w stanie obs�ugiwa� je w czasie rzeczywistym. W odr�nieniu od np. relacyjnych baz danych, nie s� to systemy o ustandaryzowanym interfejsie ani architekturze, brakuje te� podstawowych narz�dzi dla korzystaj�cego z nich analityka. Pierwszym celem tej pracy jest przegl�d istniej�cych system�w do przetwarzania strumieni danych, analiza architektury rozwi�za� dost�pnych na zasadach Open Source oraz por�wnanie ich. Drugim celem jest zaproponowanie mo�liwie uniwersalnego rozwi�zania umo�liwiaj�cego interaktywn� prac� ze strumieniem, tzn. definiowanie agregat i ogl�danie ich dynamiczne zmieniaj�cych si� wynik�w w kr�tkich cyklach.
\end{abstract}

\tableofcontents
%\listoffigures
%\listoftables

\chapter*{Wprowadzenie}
\addcontentsline{toc}{chapter}{Wprowadzenie}

TODO

\chapter{Podstawowe poj�cia}\label{r:pojecia}

TODO

\section{Definicje}

TODO
\begin{defi}\label{strumie�}
  TODO
\end{defi}

\begin{thebibliography}{99}
\addcontentsline{toc}{chapter}{Bibliografia}
\end{thebibliography}

\end{document}


%%% Local Variables:
%%% mode: latex
%%% TeX-master: t
%%% coding: latin-2
%%% End:
